% reese_ex2.tex - Report for first assignment for PPLS

\documentclass[10pt]{report}
\usepackage{graphicx}
\usepackage{cite}
\usepackage{url}
\usepackage[small,labelfont=it,textfont=it]{caption}
\usepackage{subfig}
\usepackage{listings}
\lstset{
  numbers=left,
  captionpos=b
}
\usepackage{longtable}
\setcounter{secnumdepth}{0}
\begin{document}
\title{Exercise 2 - PPLS\\Spring 2012}
\author{Josh Reese\\s1129936}
\maketitle
\section{BSP}
The Bulk Synchronous Parallel (BSP) model is a model for parallel
programming that allows the developer to abstract themselves from the
specific parallel architecture they are using \cite{bsp}. In other words this
model provides a programming framework for which software can be
developed and run on shared memory systems, message passing systems,
and various other types of parallel architectures. BSP claims three
important fundamental properties which are: simple to write,
architecture independence, and the performance of a program is
\emph{predictable} \cite{bsp}. To accomplish this the model looks at a program
as a whole, rather than as individual processes, and determines what they
call the \emph{bulk} properties of a program. To simplify and make
this possible BSP makes no attempt to use locality as a performance
optimization. This means that concepts such as cache efficiency and data
locality play no role in the BSP model.

The BSP programming style is broken into three main phases (called a
\emph{superstep}): local
computations (values in each processes local memory), communication
between processes, and a synchronization barrier \cite{bsp}. In this style the
interaction between processes occurs during the communication phase
and all the processes are synchronized on the barrier at the end of
the superstep.

To determine the performance cost BSP uses four parameters: $h$,
$g$, $l$, and $w$. In every communication phase each process will be
sending and receiving a fixed number of messages the maximum of which
is represented by $h$ and known as an $h$-relation. This term also has
the size of the message folded into it. The parameter $g$
represents the physical ability of the network to deliver messages.
This is defined such that it takes time $hg$ to deliver an
$h$-relation. The value of $l$ is determined empirically and is taken
to be the cost of the barrier. The last parameter, $w$, is the time to
process a local computation. Putting this all together yields:\\
\[ cost\: of\: a\: superstep\: =\; \stackrel{MAX}{processes}\: w_{i}\: +\:
\stackrel{MAX}{processes}\:h_{i}g\: +\: l\]
Where $i$ is the range over all processes \cite{bsp}.

When considering $pipelined$ parallel computations BSP seems to have a
few issues. The main concern this model appears to have with
pipelined processing is that it is limited by its barrier stage. In
a standard pipeline computation whenever a process is finished with
its data it is free to send data, receive data, and start working on
any new data whenever it is ready. However, in
the BSP model nobody is free to do anything with its received data
until the last process has finished and thus the superstep has passed
the barrier. This limits the pipeline process because each
process in the BSP model will need to have a balanced workload
otherwise the pattern will breakdown because all the processes will be
waiting on one or more slow processes. The concept inherent in BSP in
the global barrier breaks one of the main goals of pipelined
computations which is that each process is basically an atomic unit
which only depends on its predecessor.

Similarly, the \emph{bag of tasks} computations are likely to cause
some issues in the BSP model. Since this model relies on
communications between one process and all the other processes the way
in which the BSP model handles communication will cause a serious
problem for this pattern. Since all the processes will be trying to
communicate with a single farmer there will likely be large floods of
communication going to one process at one time. Conversely, if the
farmer has received a large number of messages that process will
likely have to send out a large amount of message directly
afterwards. This puts heavy amounts of communication on one node in
the process pool. In addition to this the BSP model breaks the
fundamental concept of a bag of tasks which is that the workers should
be completely independent from one another. The only similarity
workers should have is that they communicate with the farmer. However,
the BSP model forces the workers, and farmer, into sending messages
simultaneously, and working in tandem as opposed to independently. 


\bibliography{reese_ex2}{}
\bibliographystyle{plain}
\end{document}
