% reese_ex1.tex - Report for first assignment for PPLS

\documentclass[10pt]{report}
%% \usepackage[margin=2.5cm]{geometry}
\usepackage{graphicx}
\usepackage{cite}
\usepackage{url}
\usepackage[small,labelfont=it,textfont=it]{caption}
\usepackage{subfig}
\usepackage{listings}
\lstset{
  numbers=left,
  captionpos=b
}
\usepackage{longtable}
\setcounter{secnumdepth}{0}
\begin{document}
%% \title{Exercise 1 - PPLS\\Spring 2012}
%% \author{Josh Reese\\s1129936}
%% \maketitle
\section{1}
This algorithm is modeled very closely on the Test-and-Set algorithm
presented in the slides. We can see that this will guarantee mutual
exclusion by examining the worst case scenario: multiple threads reach
the while loop on line 4 at the same time while the shared lock
variable (L) appears to be free. These multiple threads all
recognize that the lock is greater than 0 so they proceed to enter
the DEC operation. However, the DEC operation is an atomic action so
only one thread will be allowed to access this at a time. The first
thread (thread A) to get in will decrement the value of the shared L variable
from 1 (which is the highest value L can take and is the value which
denotes the lock is open) to 0 and will return with a value of false. Thus
allowing this thread to proceed to the critical section. The next
thread (and all subsequent threads that made it to the DEC operation)
will decrement the value of the shared lock bringing the value to a
number below 0 and will thus return true and not be allowed to enter
the critical section. Once thread A has finished its critical section
it sets the lock to one (indicating the lock is free) and continues
with its tasks. To produce a more cache efficient algorithm I have
implemented a strategy similar to the Test-and-Test-and-Set using the
lazy OR operator. For a thread to be accessing its critical section
the lock must be less than or equal to zero. So, when a thread is
checking to see if it is free to enter the critical section it must
first make sure L is larger than 0. By adding the first clause in the
while loop found on line 4 we ensure that threads are only writing to
the shared variable, and thus updating cache less frequently, when
they have checked to see if it is possible for them to enter the
critical section.
\begin{lstlisting}[caption={Pseudo code solution for critical section problem
    using the DEC operation.},label={fig:code0}]
int L = 1;
co [i = 1 to n] {
  while (something) {
    while(L<1 || DEC(L)) ; ##check the lock first, then do DEC
                           ##if the lock appears free
    critical section;
    L = 1;                 ##open the lock
    non-critical;
   }
}
oc
\end{lstlisting}
\newpage
\section{2}
In this version of the DLF algorithm each node will be maintaining the
following parameters locally:
\begin{list}{-}{}
  \item ID number (unique to each node).
  \item list of IDs of all connected nodes: neighbors[].
  \item palette of forbidden colors that have been used by its
    neighbors: usedcolor($v$) and is initially empty.
\end{list}
In shared memory the following will be stored:
\begin{list}{-}{}
  \item A structure for each node which will contain: the nodes
    degree [deg($n$)], random value generated each round [rndvalue($n$)],
    and its proposed color for the round. This will be referenced by
    the ID of the node.
\end{list}
Similar to the algorithm presented in the paper this algorithm will be
working in rounds where in each round the nodes are trying to obtain a
color. Ties are broken the same as is discussed in the paper where for
neighboring nodes $n_{1},n_{2}\ \in\ V$ the priority of $n_{1}$ is
higher if:
\begin{center}
  {deg(ID($n_{1}$)) $>$ deg(ID($n_{2}$))}
\end{center}
or
\begin{center}
  {(deg(ID($n_{1}$)) == deg(ID($n_{2}$))) and
  (rndvalue(ID($n_{1}$)) $>$ rndvalue(ID($n_{2}$)))}
\end{center}
Where ID($n_x$) denotes node $x$'s unique ID.\\

In each round of the algorithm each uncolored node $n$ does the
following:
\begin{enumerate}
  \item Choose a random number uniformly distributed on [0..1] and set
    the corresponding value in its shared structure.
  \item Choose the first legal color (not in node $n$'s forbidden
    colors list) and set the corresponding element in its structure.
  \item At this point there is a synchronization barrier which all
    processes must reach before continuation.
  \item Compare its local parameters to those of each of its neighbors
    (referenced by neighbors[]) to
    determine if it has the highest priority.
  \item If node $n$ doesn't have the highest priority among its neighbors, or
    its proposed value clashes with its neighbors proposals update the
    list usedcolors($n$). Otherwise this node is done and the value
    stored as its proposed value is its final color.
  \item At this point there is a synchronization barrier which all
    processes must reach before continuation.
\end{enumerate}

The pattern in this algorithm resembles an interacting peers pattern
in that everyone is executing the same code but on different portions
of the data. There is also some similarity to the interacting peers
example discussed in lectures, the Finite Difference
Method - Jacobi, where some code is
executed followed by a barrier followed by some code then a
barrier. Due to the nature of the program several barriers are
required to keep the different processes synchronized and accessing
the correct information.

If we were to consider the case where we had
more nodes in the graph than processors it is clear each processor
will have to deal with multiple nodes and this could cause a loss in
parallelism in the previously proposed algorithm. In the previous
algorithm once a node is colored in it stops working. If we imagine
initializing the workload to be even across all processors (in that
each processor is handling the same amount of nodes) unless we
redistribute the workload as nodes become filled in some processors
will inevitably have less work to do than others. In this case it
might make more sense to organize this into a variant of a bag of
tasks problem with multiple
bags and access to each bag is synchronized by the barriers in the
algorithm.  

\end{document}
